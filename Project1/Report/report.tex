% !TEX encoding = UTF-8 Unicode
% !TEX program = latexmk
% !TEX options = -xelatex -synctex=1 -shell-escape -interaction=nonstopmode -file-line-error -output-directory=out
% !TEX root = ./CT2.tex
%----------------------------------------------------------------------------------------
% Coursework/Project Requirements
% What is the mark scheme?
% Here is the thought experiment.
% Consider an external examiner (a professor from another UK
% university).
% They have heard of Lean and that know it’s a “theorem prover”.
% They have no idea how it works.
% What would they think of your project?
% Would they be able to understand something about what you
% are doing?
% Would they think “this is pretty cool, we should be teaching this
% at my university”?

% For the first project. I am looking for:
% ▶ some Lean code (150–200 lines of code including
% comments? More if you like?)
% ▶ A pdf explaining what is going on (5 pages? More if you
% like?)
% You won’t get penalised for doing too much (but it will take
% longer).
% I’m expecting about 15 (or more) hours of work.
%----------------------------------------------------------------------------------------
% PACKAGES AND OTHER DOCUMENT CONFIGURATIONS
%----------------------------------------------------------------------------------------

\documentclass[
  % UTF8, % Encoding %not needed
  12pt, % Default font size, values between 10pt-12pt are allowed
  % letterpaper, % Uncomment for US letter paper size, the default paper size is a4paper
  % Chinese, % Uncomment for Chinese document
]{assignment}

\usepackage{amsthm} % For proof environment
\newtheorem{lemma}{Lemma} % define lemma environment
% \usepackage{pdfpages} % 用于插入 PDF
\usepackage{fontspec} % XeLaTeX 或 LuaLaTeX 使用 fontspec
% \usepackage{hyperref}
% \usepackage{embedfile}
% \usepackage{appendix}
\usepackage{mathtools}
\newcommand\nb{\addtocounter{equation}{1}\tag{\theequation}}
\pgfplotsset{compat=1.18}
%
% \renewcommand{\thesection}{Exercise \arabic{section}}

%

%-------------------------------------------------------------------------------
% ASSIGNMENT INFORMATION
%-------------------------------------------------------------------------------

\title{Coursework 1: Group Actions} % Assignment title

\author{Frankie Feng-Cheng WANG} % Student name

% \studentid{219046894} % Student id Leicester


\date{\today} % Due date

\institute{Department of Mathematics\\Imperial College London} % Institute or school name

\course{MATH70040-Formalising Mathematics} % Course or class name

\lecturer{Dr Bhavik Mehta} % Lecturer or teacher in charge of the assignment


%----------------------------------------------------------------------------------------
\begin{document}
\maketitle % Output the assignment title, created automatically using the information in the custom commands above
\tableofcontents

%----------------------------------------------------------------------------------------
% ASSIGNMENT CONTENT
%----------------------------------------------------------------------------------------
\section{Introduction}
This is my submission for Coursework 1 of MATH70040-Formalising Mathematics, on the topic of Group Actions. In this project, I formalise the definition of group actions, and prove some basic properties about them using Lean.
The Main Theorm is
\begin{quote}
\textbf{Theorem:} Let \(X\) be a \(G\)-Set, for each \(g \in G\), the map \(\varphi_g : X \to X\) defined by \(\varphi_g(x) = g \cdot x\) is a permutation of \(X\). Also the map \(\Phi : G \to S_X\) defined by \(\Phi(g) = \varphi_g\) is a group homomorphism with the property that for all \(g \in G\) and \(x \in X\), \(\Phi(g)(x) = g \cdot x\).
\end{quote}
% The theorm is come from the textbook written by Fraleigh and Katz \cite{fraleigh2003first}.
% John B. Fraleigh, Victor J. Katz, *A First Course in Abstract Algebra*,
% Addison–Wesley, 2003, Section 16 (Group Actions).
\section{Definitions}
In this section, I define the basic notions of group actions, including the definition of a group action, and some examples of group actions.
\subsection{Lean Code}
\begin{minted}[frame=lines,
  framesep=2mm,
  baselinestretch=1.2,
  fontsize=\footnotesize,
  linenos,
  % numbers=right,
  ]{lean}
/-! ## Definitions -/
-- Group action
class GroupAction (G : Type*) [Monoid G] (X : Type*) where
  act : G → X → X
  ga1 : ∀ (g1 g2 : G) (x : X), act (g1 * g2) x = act g1 (act g2 x)
  ga2 : ∀ (x : X), act (1 : G) x = x

variable {G : Type*} [Group G] {X : Type*} [GroupAction G X]
/-!
\end{minted}
\section{Main Theorem}

%Reference Section




\end{document}
