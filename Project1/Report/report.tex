% !TEX encoding = UTF-8 Unicode
% !TEX program = latexmk
% !TEX options = -xelatex -synctex=1 -shell-escape -interaction=nonstopmode -file-line-error -output-directory=out
% !TEX root = ./CT2.tex

%----------------------------------------------------------------------------------------
% PACKAGES AND OTHER DOCUMENT CONFIGURATIONS
%----------------------------------------------------------------------------------------

\documentclass[
  % UTF8, % Encoding %not needed
  12pt, % Default font size, values between 10pt-12pt are allowed
  % letterpaper, % Uncomment for US letter paper size, the default paper size is a4paper
  % Chinese, % Uncomment for Chinese document
]{assignment}

\usepackage{amsthm} % For proof environment
\newtheorem{lemma}{Lemma} % define lemma environment
% \usepackage{pdfpages} % 用于插入 PDF
\usepackage{fontspec} % XeLaTeX 或 LuaLaTeX 使用 fontspec
% \usepackage{hyperref}
% \usepackage{embedfile}
% \usepackage{appendix}
\usepackage{mathtools}
\newcommand\nb{\addtocounter{equation}{1}\tag{\theequation}}
\pgfplotsset{compat=1.18}
%
\renewcommand{\thesection}{Exercise \arabic{section}}

%

%-------------------------------------------------------------------------------
% ASSIGNMENT INFORMATION
%-------------------------------------------------------------------------------

\title{Coursework 2} % Assignment title

\author{Frankie Feng-Cheng WANG} % Student name

% \studentid{219046894} % Student id


\date{\today} % Due date

\institute{Department of Mathematics\\Imperial College London} % Institute or school name

\course{MATH70061 - Commutative Algebra} % Course or class name

\lecturer{Prof Paolo Cascini} % Lecturer or teacher in charge of the assignment

%----------------------------------------------------------------------------------------
\begin{document}
\maketitle % Output the assignment title, created automatically using the information in the custom commands above
% \tableofcontents

%----------------------------------------------------------------------------------------
% ASSIGNMENT CONTENT
%----------------------------------------------------------------------------------------
\section{}
Let k be a field.
\subsection{}
Prove that for any subsets X,Y of \(\mathbb{A}^N_k\), we have \(I(X\cup Y)= I(X) \cap I(Y)\)
\begin{proof}

\begin{align*}
    &f \in I(X\cup Y)\\
    \Leftrightarrow &\forall z  \in X \cup Y , f(z)=0\\
    \Leftrightarrow &\forall z  \in X  , f(z)=0 \And \forall z  \in  Y , f(z)=0\\
    \Leftrightarrow & f \in I(X) \And f \in  I(Y)\\
    \Leftrightarrow & f \in I(X) \cap I(Y)
\end{align*}
Hence \[
\boxed{I(X\cup Y)= I(X) \cap I(Y)}
\]

\end{proof}
%-------------------------------------------------------------------------------
\subsection{}Let $f, g \in k[x, y]$. Prove the identity
\[
\sqrt{(fg)}= \sqrt{(f)} \cap \sqrt{(g)}
\]
\begin{proof}


\begin{align*}
&h \in \sqrt{(fg)}\\
\Leftrightarrow &\exists n > 0 \quad \st \quad h^n \in (fg) \nb \label{eq:1}\\
\Leftrightarrow &\exists n > 0, \exists j \in k[x,y] \st h^n= jfg \\
\text{Since}&\quad  jf \in k[x,y] \And jg \in k[x,y]\\
\Rightarrow& \exists n > 0 \quad \st \quad h^n \in (f) \And h^n \in (g) \nb \label{eq:2}\\
\Leftrightarrow& h \in \sqrt{(f)} \And h \in \sqrt{(g)}\\
\Leftrightarrow& h \in \sqrt{(f)} \cap \sqrt{(g)}
\end{align*}

claim: \ref{eq:2}$\Rightarrow$\ref{eq:1}
\begin{align*}
    \ref{eq:2}
    \Rightarrow& \exists n >0,\; \exists k_1 , k_2 \in k[x,y]  \st h^n =k_1 f= k_2 g\\
    \Rightarrow& \exists n >0,\; \exists k_1 , k_2 \in k[x,y]  \st h^{2n} =k_1 k_2 fg\\
    \Rightarrow& \exists n >0\; \st \; h^{2n}\in (fg)\\
    \Leftrightarrow& \ref{eq:1}
\end{align*}
Hence we have \(h \in \sqrt{(fg)}\Leftrightarrow h \in \sqrt{(f)} \cap \sqrt{(g)}\), that is
\[
\boxed{
\sqrt{(fg)}= \sqrt{(f)} \cap \sqrt{(g)}
}
\]
\end{proof}
%-------------------------------------------------------------------------------
\subsection{}
% show that (xm) ∩(yn) = (xmyn).
show that \((x^m) \cap (y^n) = (x^m y^n)\).
\begin{proof}
\begin{align*}
    &z \in (x^m) \cap (y^n)\\
    \Leftrightarrow& \exists k_1 , k_2 \in k[x,y] \; \st \; z=k_1 x^m= k_2 y^n \nb \label{eq:1.3.1}\\
\end{align*}
Since \( x^m \nmid y^n\) and \( y^n \nmid x^m\), we have
\(y^n \mid k_1\) and \(x^m \mid k_2\). Thus there exists \(k_3 \in k[x,y]\) such that
\begin{align*}
    &k_1 = k_3 y^n
\end{align*}

Substituting into \ref{eq:1.3.1}, we have
\begin{align*}
    &\exists k_3 \in k[x,y] \quad \st \quad z = k_3 y^n x^m \nb\label{eq:1.3.2}\\
    \Leftrightarrow& z \in (x^m y^n)
\end{align*}
Hence we have
\[(x^m) \cap (y^n) \subseteq (x^m y^n)\]
% Remark: 1.3.3 到1.3.4的推导过程可逆,见下文。 但是我们也可以直接证明另一半包含关系,如下。
Remark: The argument from \ref{eq:1.3.1} to \ref{eq:1.3.2} is reversible. However, we can also directly prove the other inclusion, as follows.

Claim: \((x^m) \cap (y^n) \supseteq (x^m y^n)\)\\
For any \(z \in (x^m y^n)\), there exists \(k_3 \in k[x,y]\) such that \( z=k_3 y^n x^m \).
Consiering the above argument backwards, we have
 \[k_3y^n \in k[x,y] \Rightarrow z \in (x^m)\]
  and \[k_3x^m \in k[x,y] \Rightarrow z \in (y^n)\].
\[\Rightarrow z \in (x^m) \cap (y^n)\]
That is, \((x^m) \cap (y^n) \supseteq (x^m y^n)\).
Hence we have
\[
\boxed{(x^m) \cap (y^n) = (x^m y^n)}
\]
\end{proof}
%----------------------------------------------------------------------------------------
\section{}
% Let R= Z/60Z. Compute the nilradical N (R) and the
% Jacobson radical J(R).
Let \(R= \mathbb{Z}/60\mathbb{Z}\). Compute the nilradical \(N (R)\) and the
Jacobson radical \(J(R)\).
\begin{proof}
  Since \(R=\mathbb{Z}/60\mathbb{Z}\sim \mathbb{Z}_{60}\sim \mathbb{Z}_{4}\times \mathbb{Z}_{3} \times \mathbb{Z}_{5}\).\\
  w.l.o.g, we consider \(R=\mathbb{Z}_{4}\times \mathbb{Z}_{3} \times \mathbb{Z}_{5}\).\\
  For \(\mathbb{Z}_{4}\), the nilradical is \((2)\) and the Jacobson radical is \((2)\).\\
  \begin{itemize}
    \item Nilradical: since \(2^2 \equiv 0 \mod 4\), we have \((2)\) is nilpotent ideal. And the only ideals of \(\mathbb{Z}_{4}\) are \((0),(2),(1)\). Thus the nilradical is \((2)\).
    \item Jacobson radical: since \(\mathbb{Z}_{4}\) has a unique maximal ideal \((2)\), we have the Jacobson radical is \((2)\).
  \end{itemize}
  For \(\mathbb{Z}_{3}\), the nilradical is \((0)\) and the Jacobson radical is \((0)\).\\
  \begin{itemize}
    \item Nilradical: since \(\mathbb{Z}_{3}\) is a field, it has no non-zero nilpotent elements (see lemma \ref{lemma:2.1}). Thus the nilradical is \((0)\).
    \item Jacobson radical: since \(\mathbb{Z}_{3}\) is a field, it has no maximal ideals except \((0)\) (No nontrivial proper ideals). Thus the Jacobson radical is \((0)\).
  \end{itemize}
  For \(\mathbb{Z}_{5}\), the nilradical is \((0)\) and the Jacobson radical is \((0)\).(Same argument as for \(\mathbb{Z}_{3}\))\\
  % \begin{itemize}
  %   \item Nilradical: since \(\mathbb{Z}_{5}\) is a field, it has no non-zero nilpotent elements. Thus the nilradical is \((0)\).
  %   \item Jacobson radical: since \(\mathbb{Z}_{5}\) is a field, it has no maximal ideals except \((0)\). Thus the Jacobson radical is \((0)\).
  % \end{itemize}
  Thus we have
  \[N(R)=(2)\times (0) \times (0)=\{(0,0,0),(2,0,0)\}\]
  and
  \[J(R)=(2)\times (0) \times (0)=\{(0,0,0),(2,0,0)\}\]
  Hence we have
  \[\boxed{N(R)=J(R)=\{(0,0,0),(2,0,0)\}}\]
Bringing back to \(R=\mathbb{Z}/60\mathbb{Z}\), we have
\[[30]^2 =30^2+60\mathbb{Z}=900+60\mathbb{Z}=0+60\mathbb{Z}=60\mathbb{Z}=[0]\]
\[\boxed{N(R)=J(R)=\{ [0],[30]\}}=([30])\]
\end{proof}
\begin{lemma}
\label{lemma:2.1}
  A field has no non-zero nilpotent elements.
\begin{proof}
  Let \(F\) be a field. For any \(a \in F\), if \(a\) is non-zero nilpotent element, then \(a\) has a inverse \(a^{-1}\) and for some \(n>0\) , \(a^n=0\),we have
  \[1=aa\dots a a^{-1}a^{-1}\dots a^{-1}=a^n a^{-n}=0 a^{-n}=0\]
  which is a contradiction. Thus \(a\) must be zero or non-nilpotent.
\end{proof}
\end{lemma}
\section{}
% Let R= Z(5) be the localisation of Z at the prime ideal (5).
% Prove that every ideal of R is of the form (5n) for a unique n ≥0.
Let \(R= \mathbb{Z}_{(5)}\) be the localisation of \(\mathbb{Z}\) at the prime ideal \((5)\).
Prove that every ideal of \(R\) is of the form \((5^n)\) for a unique \(n \geq 0\).
\begin{proof}
  Local ring \(R\) at the prime ideal \((5)\) is given by
  \[
  R=\mathbb{Z}_{(5)}=\{ \frac{a}{b}\in \mathbb{Q}=Frac(\mathbb{Z}) \mid a \in \mathbb{Z}, b \in \mathbb{Z}\setminus (5)\}
  \]
  \[
  m_{(5)}= \{ \frac{a}{b} \in \mathbb{Q} \mid a \in (5), b \in \mathbb{Z}\setminus (5)\}
  \]
  % Since \(\mathbb{Z}\) is a PID, every ideal of \(\mathbb{Z}\) is of the form \((m)\) for some \(m \in \mathbb{Z}\).\\
  Suppose \(I\) is an ideal of \(R\). Consider the contraction of \(I\) in \(\mathbb{Z}\):
  \[I^c= I \cap \mathbb{Z}
  \]
  Claim: \(I^c\) is an ideal of \(\mathbb{Z}\).\\
   For any \(r \in \mathbb{Z}\) and \(a \in I^c\), since \(I\) is an ideal of \(R\), we have \(ra \in I\). Since \(r,a \in \mathbb{Z}\), we have \(ra \in \mathbb{Z}\). Thus \(ra \in I^c\). Hence \(I^c\) is an ideal of \(\mathbb{Z}\).\\
  Since \(\mathbb{Z}\) is a PID, there exists \(m \in \mathbb{Z}\) such that
  \[I^c = (m)
  \]
  Since \(I^c \subseteq (5)\) (because \(I \subseteq m_{(5)}\)), we have \(5 \mid m\). Thus there exists \(n \geq 0\) such that \(m=5^n\). Hence we have
  \[I^c = (5^n)=\{z5^n \mid z \in \mathbb{Z}\} \quad \text{for some } n \geq 0.
  \footnote{\((5^n)\) above is an ideal of \(\mathbb{Z}\), not of \(R\), and we use \((5^n)R\) to denote the ideal of \(R\) generated by \(5^n\).}
  \]
  % NOTE: \((5^n)\) above is an ideal of \(\mathbb{Z}\), not of \(R\), and we use \((5^n)R\) to denote the ideal of \(R\) generated by \((5^n)\).\\
  Claim: \(I=I^c R= (5^n)R=\{ \frac{5^n k}{s} \mid k \in \mathbb{Z}, s \in \mathbb{Z}\setminus (5)\}\) for the a \(n \geq 0\).\\
  For any \(x \in I\), there exists \(s \in \mathbb{Z}\setminus (5)\) such that \(sx \in I^c\). Thus there exists \(k \in \mathbb{Z}\) such that
  \[sx=5^n k\]
  \[\Rightarrow x=\frac{5^n k}{s} \in (5^n)R\]
  Hence we have \(I \subseteq (5^n)R\).\\
  For any \(y \in (5^n)R\), there exists \(k \in \mathbb{Z}\) and \(s \in \mathbb{Z}\setminus (5)\) such that
  \[y=\frac{5^n k}{s}\]
  Since \(5^n k \in I^c \subseteq I\) and \(\frac{1}{s} \in R\), we have
  \[y=\frac{5^n k}{s} \in I\]
  Hence we have \((5^n)R \subseteq I\).\\
  Thus we have \(I=(5^n)R\).\\
  Uniqueness: Suppose there exists \(n\neq m \geq 0\) such that \(I=(5^m)R=(5^n)R\). w.l.o.g, we assume \(m<n\). Then we have
  \[5^m \in (5^n)R\]
  \[\Rightarrow \exists k \in \mathbb{Z}, s \in \mathbb{Z}\setminus (5) \st 5^m = \frac{5^n k}{s}\]
  \[\Rightarrow  s = k5^{n-m}\]
Consider \(s \nmid 5\), but the right hand side is divisible by \(5\)(Since \(n-m\geq 1\)), which is a contradiction. Thus \( m\) must be equal to \( n\).\\
  %回到题目的记号
  Back to the notation of the question, we have
  \[
  \boxed{\forall I \text{ ideal of } R, I = (5^n)=\{ \frac{5^n k}{s} \mid k \in \mathbb{Z}, s \in \mathbb{Z}\setminus (5)\} \text{ for a unique } n \geq 0}
  \]
\end{proof}
\end{document}
